\chapter{Technologie użyte w projekcie}
System stworzony na potrzeby niniejszej pracy składa się z dwóch komponentów: warstwy logiki (ang. \textit{backend})  oraz warstwy interfejsu użytkownika (ang. \textit{frontend}).
\section{Języki C\# i XAML}
Dwoma najpopularniejszymi językami wykorzystywanymi do tworzenia warstwy logiki są C\# i Java. Oba te języki posiadają bogaty zasób narzędzi i bibliotek ułatwiających ten proces. C\# i Java znajdują również zastosowanie w dziedzinie aplikacji mobilnych, gdzie popularnie wykorzystywanymi językami są też Kotlin, Swift oraz JavaScript.

\textbf{C\#} jest prostym, nowoczesnym i zorientowanym obiektowo językiem programowania, wywodzącym się z języka C. Posiada on szereg funkcjonalności usprawniających proces wytwarzania oprogramowania, takimi jak: odśmiecanie (ang. \textit{garbage collection}), bezpieczeństwo typologiczne (ang. \textit{type safety}) czy obsługa wyjątków (ang. \textit{exception handling}).\cite{wagner_wenzel_latham_onderka_2016}\\
Język C\# został użyty do stworzenia warstwy logiki oraz części warstwy interfejsu użytkownika odpowiedzialnej za komunikację z warstwą logiki. Wybór ten uzasadniony jest dostępnymi dla tego języka narzędziami oraz bibliotekami.

\textbf{XAML (Extensible Application Markup Language)} jest deklaratywnym językiem stworzonym na bazie języka XML. Pozwala on na definiowanie interfejsu użytkownika przy użyciu znaczników, podczas gdy zachowanie poszczególnych komponentów interfejsu opisane jest za pomocą kodu w języku C\#.\cite{britch_dunn_schwelnus_2018} \\
Język XAML świetnie sprawdza się w opisie interfejsu użytkownika aplikacji stworzonej przy użyciu C\#, dlatego został on użyty do stworzenia aplikacji mobilnej.
\section{Platformy programistyczne (frameworki)}
%TODO framework def
Obecnie na rynku istnieje wiele technologii umożliwiających tworzenie takich systemów, np. \textit{Spring, ASP .NET, ASP .NET} do warstwy logiki oraz \textit{Xamarin Native, Xamarin Forms, React Native} do aplikacji mobilnych.
Warstwa logiki powstała przy użyciu platformy \textbf{.NET Core}. Jest to otwartoźródłowa platforma programistyczna ogólnego użytku, wyróżniająca się kompatybilnością z wieloma systemami operacyjnymi oraz architekturami. Z platformy tej można korzystać używając języków C\#, F\# oraz Visual Basic.
\section{Narzędzia}

%TODO: MVC, REST, SWAGGER, LIBRARIES, EF CORE