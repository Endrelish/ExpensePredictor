\chapter{Wstęp}
\section{Problematyka i zakres pracy}
Niniejsza praca porusza problem zarządzania budżetem oraz predykcji wydatków, czyli ich prognozowania. Istnieje wiele rozwiązań służących do rejestracji przychodów i wydatków, zarówno w postaci systemów komputerowych, jak i tradycyjnych metod. Systemy, z których korzysta wielu użytkowników przechowują dane o dochodach i wydatkach użytkowników. Zbiór takich danych daje możliwość wykorzystania ich do modelowania zależności między nimi i wykonywania predykcji na ich podstawie.

Prognozowanie wydatków może w znaczącym stopniu pomóc w planowaniu budżetu oraz ułatwić zarządzanie dostępnymi środkami pieniężnymi.

W zakres niniejszej pracy wchodzi stworzenie systemu, który umożliwia rejestrację wydatków i przychodów użytkowników, a następnie na podstawie tak zebranych danych zapewnia prognozowanie przyszłych wydatków. System ten składa się z części serwerowej odpowiedzialnej za wykonywanie wszelkich operacji na zebranych danych oraz aplikacji klienckiej, która zapewnia interakcję systemu z użytkownikiem.
\section{Cele pracy}
Niniejsza praca powstała w celu:
\begin{itemize}
	\item wyboru odpowiednich metod analizy danych, zastosowania ich w aplikacji służącej do predykcji wydatków oraz przeglądu istniejących aplikacji oferujących zbliżone funkcjonalności,
	\item stworzenia aplikacji mobilnej służącej do rejestrowania i planowania wydatków, wykorzystującej zebrane w ten sposób dane statystyczne do ich predykcji,
	\item określenia użyteczności stworzonej aplikacji, przydatności metod analizy danych do predykcji wydatków oraz porównania różnych modeli reprezentujących przetwarzane dane.
\end{itemize}
\section{Przegląd literatury w dziedzinie komputerowej analizy danych}
\textbf{Beata Pułaska-Turyna, Statystyka dla ekonomistów. Wydanie III zmienione.} - pozycja ta opisuje metody służące do analizy i modelowania struktury danych oraz ich szacowania i prognozowania.

\textbf{Ryan Tibshirani, Nonparametric Regression} - praca zawierająca opis metod regresji nieparametrycznej oraz wyznaczania jej estymatorów.
\section{Opis zawartości pracy}
W drugim rozdziale niniejszej pracy opisane zostały istniejące rozwiązania w dziedzinie zarządzania budżetem oraz metody analizy danych służące do predykcji. Trzeci rozdział zawiera opis technologii użytych w projekcie systemu ExpensePredictor. Dokumentacja techniczna systemu ExpensePredictor zawarta jest w rozdziale czwartym, a w rozdziale piątym znajduje się dokumentacja użytkownika tego systemu.