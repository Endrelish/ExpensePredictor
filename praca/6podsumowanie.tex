\chapter{Podsumowanie}
\section{Wnioski}
Stworzony system ExpensePredictor jest systemem służącym do zarządzania budżetem. Na rynku istnieje wiele produktów o takim przeznaczeniu. System ExpensePredictor wyróżnia się tym, że oferuje funkcjonalność unikatową, którą jest predykcja wydatków, czyli przewidywanie wartości przyszłych wydatków użytkownika.

Predykcja danych opiera się na metodach regresji parametrycznej lub nieparametrycznej. Regresja parametryczna jest w stanie stworzyć matematyczny model danych i na jego podstawie wykonywać prognozę, lecz wymaga ściśle ustalonej struktury danych i możliwości matematycznego opisu zależności między nimi. Regresja nieparametryczna sprawdza się lepiej w przypadkach, gdy zależności między danymi nie są ściśle określone i nie jest możliwe stworzenie matematycznego modelu je opisującego.

Kategoryzacja wydatków pozwala uwzględnić różnice w zależnościach między danymi o wydatkach z różnych kategorii. Stworzenie modelu regresji dla konkretnej kategorii danych pozwala na dokładniejsze modelowanie zależności, a co za tym idzie trafniejsze prognozy.

Dzięki uwzględnieniu danych wszystkich użytkowników systemu w procesie tworzenia modelu regresji możliwa jest analiza większego przekroju danych statystycznych, co przekłada się na bardziej dopasowany do rzeczywistości model regresji.
\section{Perspektywy rozwoju}
Po zebraniu i analizie stosunkowo dużego zbioru danych użytkowników systemu możliwa będzie modyfikacja istniejącej bądź stworzenie własnej metody analizy danych, która lepiej odwzoruje zależności między zebranymi danymi i pozwoli na jeszcze trafniejsze prognozy.

W celu ułatwienia i usprawnienia procesu korzystania z systemu, będzie w przyszłości możliwa integracja z systemem e-paragonów. System ten, według Ministerstwa Cyfryzacji, ma zostać wprowadzony w Polsce i do końca 2020 roku ma w całości zastąpić fizyczne paragony.\cite{eparagon} Możliwa jest również integracja z istniejącymi systemami bankowymi i rejestracja wydatków i przychodów na podstawie danych udostępnionych przez te systemy.

Integrując system ExpensePredictor z zewnętrznymi systemami dostarczającymi dane o wydatkach i przychodach możliwe będzie opracowanie i zastosowanie algorytmu pozwalającego na automatyczną kategoryzację wydatków. Usprawni to proces korzystania z systemu przez użytkownika.

